\begin{table}[htbp]
\centering
\footnotesize
\def\sym#1{\ifmmode^{#1}\else\(^{#1}\)\fi}
\caption{First Stage Regressions for Initial Assessment of Mental Health Court (Under 25)}\label{tab:firststage}
\begin{tabular}{p{.35\linewidth}*{2}{c}}
\toprule
                    &\multicolumn{1}{c}{(1)}   &\multicolumn{1}{c}{(2)}   \\
\midrule
Z: Clinician's Leave-Out Mean Mental Health Score&       0.627***&       0.588***\\
                    &     (0.094)   &     (0.092)   \\
\midrule
Kleibergen-Paap F   &     44.2100   &     41.0831   \\
Time Fixed Effects  &         Yes   &         Yes   \\
Baseline Controls   &          No   &         Yes   \\
Observations        &       9,532   &       9,532   \\
\bottomrule
\multicolumn{3}{p{.7\textwidth}}{\tiny We report the first stage results of a linear probability model with outcome of interest being the initial assessment of an inmate's mental health being moderate to severe relative to none or mild. The propensity to assign a high score is estimated using data from other cases assigned to the clinician following the procedure described in the text. Column (1) shows the results by controlling only for day-of-week-month fixed effects, whereas Column (2) also includes the inmate baseline controls as shown in Table 1. Each column gives the corresponding clinician and inmate robust two-way clustered standard errors in parentheses. Robust (Kleibergen-Paap) first stage F reported (which is equivalent to the effective F-statistic of Montiel Olea and Pflueger (2013) in this case of a single instrument). * p$<$0.10, ** p$<$0.05, *** p$<$0.01}
\end{tabular}
\end{table}
